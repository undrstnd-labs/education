\thispagestyle{empty}
%\mtcaddchapter[Introduction générale]
\begin{center}
  \textbf{\Huge Introduction générale}
\end{center}

L'internet a transformé profondément les modes d'apprentissage et de consommation des informations éducatives. Ces dernières décennies, l'éducation à distance et l'accès aux ressources pédagogiques en ligne ont connu une croissance exponentielle.\\
En effet, les plateformes E-learning permettent aux étudiants de continuer leur apprentissage même en dehors des salles de classe traditionnelles. Les étudiants gagnent un temps considérable en ayant accès à des ressources pédagogiques en ligne à tout moment. Cette méthode d'apprentissage est particulièrement bénéfique dans le contexte actuel, où de nombreux étudiants rencontrent des obstacles à la compréhension de leurs cours en raison des interruptions fréquentes et des limitations des supports pédagogiques traditionnels. De plus, la disponibilité immédiate de ressources pédagogiques en ligne aide les étudiants à surmonter les défis liés à l'accès limité aux documents physiques. En retour, les étudiants peuvent améliorer leur compréhension des matières et progresser à leur propre rythme. Actuellement, la demande pour des solutions E-learning efficaces est en forte croissance, notamment avec l'émergence de technologies éducatives avancées. De plus, l'intelligence artificielle est maintenant présente partout et joue un rôle essentiel dans l'amélioration de l'expérience d'apprentissage en évaluant les demandes des étudiants et en fournissant des réponses précises, menant à une meilleure compréhension des sujets étudiés. Cependant, de nombreux défis persistent, tels que la personnalisation de l'apprentissage, l'intégration de ressources pédagogiques variées et la création d'une expérience utilisateur enrichissante. Face à ces défis, nous avons puisé notre inspiration dans notre propre expérience de la vie estudiantine. Notre objectif est de concevoir et de développer une plateforme interactive et intelligente appelée \textbf{« Undrstnd »}, visant à révolutionner l'expérience d'apprentissage des étudiants et des enseignants.\\
Le premier chapitre présentera le cadre général du projet, l'étude de l'existant et la planification de notre travail selon la méthode 2TUP. Le deuxième chapitre abordera les concepts de base liés à notre travail, en explorant des notions clés de l'intelligence artificielle et du cloud computing. Le troisième chapitre sera consacré à l'analyse et à la spécification des besoins fonctionnels et non fonctionnels, illustrés par des diagrammes de cas d'utilisation et des diagrammes de séquence d'analyse. Dans le quatrième chapitre, nous présenterons le modèle architectural de notre système, la conception de la base de données, la conception logicielle détaillée et le maquettage de quelques interfaces. Enfin, nous clôturerons notre rapport par le cinquième chapitre, consacré à la spécification des différents outils matériels et logiciels utilisés dans le développement de notre plateforme, à l'intégration et à l'évaluation du modèle utilisé, ainsi qu'à quelques captures d'écran du travail réalisé.
