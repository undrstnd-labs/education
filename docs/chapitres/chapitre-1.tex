\vfill 
\chapter{Étude Préliminaire et Contextualisation du Projet}
\label{chap:etude-preliminaire}
\vfill 
\minitoc
\mtcaddchapter
\vfill 

\newpage 

\section*{Introduction}
\justifying
Ce chapitre vise à exposer l`étude préliminaire de notre projet. Tout d’abord, nous présenterons le contexte général et l’entité qui nous a accueilli pour faire un stage de fin d`études. Puis, nous aborderons la problématique et l`idée générale de notre projet. Ensuite, nous entreprendrons une analyse approfondie de l`existant, mettant en évidence les avantages et les limites des solutions similaires présentées  sur le marché, afin de nous en inspirer et de retenir une solution plus raffinée. Enfin, nous conclurons ce chapitre en exposant la méthodologie de développement la plus adaptée à la réalisation de notre projet, et le diagramme de Gantt illustrant le planning général de celui-ci.
\section{Contexte général et cadre académique du projet}
\justifying
Dans cette partie, nous représenterons le contexte de notre projet, qui inclut le cadre académique, la présentation de l’organisme d’accueil, la problématique et l’idée générale de notre projet.

\subsection{Cadre académique du projet}
Notre projet, intitulé « \textit{Conception et développement d’une plateforme E-learning intégrant un système intelligent d’élucidation intelligente des ressources pédagogiques} », s’inscrit dans le cadre de la préparation d’un projet de fin d’études en vue de l’obtention du diplôme national de Licence en Sciences d’Informatique à  « \textbf{l’Institut Supérieur d’Informatique et de Mathématiques de Monastir (ISIMM)} » pour l’année universitaire 2023/2024. Le stage a été effectué au sein de la société \textbf{Globe Services Informatique GSI} durant une période de 4 mois.

\subsection{Présentation de l`entité d`accueil}
L’idée initiale de notre projet est proposée par nous-même et elle est adoptée par \textbf{STE Globe Services Informatique (GSI)}, une entreprise établie à Houmet Souk, Monastir, en Tunisie. Fondée le 20 mai 2000 en tant que société à responsabilité limitée (SARL), \textbf{GSI} se concentre principalement sur le commerce et la maintenance de matériel informatique, ainsi que sur le développement web.
\begin{figure}[ht]
    \centering
    \includegraphics[width=0.5\textwidth]{images/gsi-logo.png}
    \caption{Logo du « Globe Services Informatique (GSI) »}
    \label{fig:gsi-logo}    
\end{figure}

Les secteurs d`activités de GSI sont :
\begin{itemize}[itemsep=2pt, parsep=2pt]
    \item La vente et la maintenance de matériel informatique.
    \item Le développement web et mobile.
    \item La formation en informatique.
    \item La vente de logiciels.
\end{itemize}

Identité de l'organisation :
\begin{itemize}[itemsep=2pt, parsep=2pt]
    \item \textbf{Nom} : Globe Services Informatique (GSI).
    \item \textbf{Adresse} : 5 Rue de la République, Houmet Souk, Monastir, Tunisie.
    \item \textbf{Téléphone} : +(216) 73 447 447.
    \item \textbf{Fax} : (+216) 73 447 447.
    \item \textbf{E-mail} : \href{mailto:commercial@gsi.com.tn}{commercial@gsi.com.tn}.
    \item \textbf{Site web} : \href{https://www.gsi.com.tn/}{www.gsi.com.tn}.
\end{itemize}

\subsection{Identification de la problématique}
Dans le quotidien estudiantin et dans le déroulement ordinaire d'un cours, où l'enseignant anime la classe et les étudiants s'investissent pleinement dans l'acquisition et la compréhension des connaissances, des perturbations diverses sont fréquemment rencontrées. En effet, les étudiants se heurtent souvent à des obstacles dans la compréhension de leurs cours ce qui entrave leur progression académique.

\vspace{0.5em}
\noindent{De plus, les supports pédagogiques fournis par les enseignants tels que les cours, exercices, corrections et examens peuvent parfois se révéler insuffisants pour une assimilation complète. Par conséquent, de nombreux étudiants sont amenés à chercher d'autres ressources par eux-mêmes. Néanmoins, l’accès à ces ressources présente un autre défi incontournable. Cette situation limite considérablement leur capacité à acquérir efficacement des connaissances.}

\vspace{0.5em}
\noindent \textit{Exemple:} Le coût élevé de l’accès à certains documents et ressources pédagogiques d’intérêt constitue un obstacle financier majeur qui prive les étudiants d’outils essentiels fondamentaux à leur épanouissement éducatif.

\subsection{Idée générale du projet}
À l'origine de notre projet, nous avons puisé notre inspiration dans notre expérience de la vie estudiantine. Ainsi, nous cherchons à relever les défis auxquels nos pairs sont confrontés, tels que la compréhension de leurs cours, l’accès aux ressources pédagogiques adaptées, et le besoin de soutien académique personnalisé.

\vspace{0.5em}
\noindent{Notre idée consiste à concevoir et mettre en œuvre une plateforme interactive qui révolutionne l'expérience d'apprentissage des étudiants et des enseignants en fournissant des contenus multimédias et des outils d'intelligence artificielle. De plus, cette plateforme favorise l’apprentissage personnalisé et collaboratif, tout en demeurant une solution riche en services pertinents, intelligents et open-source.}

\vspace{0.5em}
\noindent{Nous visons à exploiter des technologies avancés de l’intelligence artificiel  (IA) et du cloud pour mettre en place une solution robuste.}

\vspace{0.5em}
\noindent{Avant de nous plonger dans l'analyse et la critique des solutions existantes sur le marché, il convient de préciser que notre projet est basé sur une idée préliminaire. Toutefois, cette idée peut être affinée et améliorée en s'inspirant des meilleures pratiques et des fonctionnalités offertes par les plateformes similaires. Dans un premier temps, nous allons donc expliquer les concepts de base liés à notre domaine de travail, afin de poser les fondations nécessaires à la compréhension de notre projet.}

\subsection{Concepts de base liés à notre projet} 
Notre projet s'appuie sur deux piliers fondamentaux que sont le cloud computing et l'intelligence artificielle (IA). Ces concepts jouent un rôle central de notre ambition de révolutionner l'expérience d'apprentissage des étudiants et des enseignants, en offrant des solutions innovantes et performantes. 

\begin{itemize}[itemsep=2pt, parsep=2pt]
    \item \textbf{Cloud computing} : Le cloud computing révolutionne la manière dont les services informatiques sont fournis et consommés. En exploitant le cloud, notre projet peut tirer parti d'une infrastructure évolutive et flexible, permettant un accès rapide et sécurisé aux ressources informatiques. Cette approche offre également une réduction des coûts opérationnels et une amélioration de l'efficacité grâce à la mise à l'échelle automatique des ressources en fonction des besoins.
    \item \textbf{Intelligence artificielle (IA)} : L'intelligence artificielle constitue le cœur de notre projet, offrant des fonctionnalités avancées telles que l'analyse de documents, la recommandation de contenu personnalisé et l'assistance virtuelle. Grâce à l'IA, notre plateforme peut comprendre, interpréter et répondre aux besoins des utilisateurs de manière intelligente, offrant ainsi une expérience d'apprentissage plus personnalisée et efficace. 
\end{itemize}

\noindent{En combinant les capacités du cloud computing et de l'IA, notre projet vise à fournir une solution innovante et intelligente pour répondre aux défis de l'éducation contemporaine.}

\section{Analyse de l'existant et perspectives critiques}
L’étude de l’existant est une étape primordiale qui permet de définir les points forts et les points faibles des systèmes similaires actuellement en place. Alors, cette section sera dédiée à faire une étude approfondie et critique des solutions existantes.                             

\subsection{Étude des solutions existantes}
Dans le domaine des technologies éducatives, plusieurs plateformes se démarquent par leur contribution à l'apprentissage. Nous étudierons ces solutions pour comprendre leurs forces. Nous identifierons également les opportunités d'amélioration que notre solution pourrait exploiter. 

\vspace{0.5em}
\noindent \textbf{Étant donné qu’il n’existe pas de solutions tunisiennes similaires à notre projet, notre revue se concentre sur le marché international.}

\vspace{0.5em}
\noindent Ainsi, nous avons porté notre attention sur les solutions étrangères les plus connues qui s'alignent avec le contexte de notre projet. Notre étude se focalise, particulièrement sur \textit{Google Classroom}, \textit{Poe}, \textit{ChatGPT} et \textit{Piazza}, les quatre solutions les plus populaires et adaptées dans le monde entier.

\begin{itemize}[itemsep=2pt, parsep=2pt]
    \item \textbf{Google Classroom} est une plateforme éducative qui permet aux enseignants de créer des salles de classe virtuelles pour leurs étudiants, partager des documents, des devoirs et communiquer avec les apprenants. Le logo de Google Classroom est illustré à la figure \ref{fig:google-classroom-logo},    
     \begin{figure}[ht]
        \centering
        \includegraphics[width=0.2\textwidth]{images/google-classroom-logo.png}
        \caption{Logo de « Google Classroom »}
        \label{fig:google-classroom-logo}
    \end{figure}
    permet de :
    \begin{itemize}[itemsep=1pt, parsep=1pt]
        \item Gérer les cours ainsi que les ressources pédagogiques pour les étudiants.
        \item Faire des discussions dans les classes.
    \end{itemize}
    \item \textbf{Piazza} est une plateforme de collaboration en ligne conçue pour faciliter la communication entre les étudiants et les professeurs. Le logo de cette plateforme est affiché dans la figure \ref{fig:piazza-logo}, permet de :
    \begin{figure}[ht]
        \centering
        \includegraphics[width=0.2\textwidth]{images/piazza-logo.png}
        \caption{Logo de « Piazza »}
        \label{fig:piazza-logo}
    \end{figure}
    \begin{itemize}[itemsep=1pt, parsep=1pt]
        \item Stocker les documents.
        \item Organiser des discussions entre les étudiants et les enseignants.
    \end{itemize}
    \item \textbf{ChatGPT} ou Chat Generative Pretrained Transformer, est un chatbot doté d'intelligence artificielle qui fournit des réponses textuelles instantanées aux questions des utilisateurs. Il se positionne comme un outil polyvalent répondant aux besoins diverse. Le logo de ChatGPT est illustré à la figure \ref{fig:chatgpt-logo}, permet de :
    \begin{figure}[ht]
        \centering
        \includegraphics[width=0.2\textwidth]{images/chatgpt-logo.png}
        \caption{Logo de « ChatGPT »}
        \label{fig:chatgpt-logo}
    \end{figure}
    \begin{itemize}[itemsep=1pt, parsep=1pt]
        \item Répondre aux questions et aux requêtes en se basant sur le contexte de la conversation.
        \item Fournir des informations sur divers sujets.
    \end{itemize}
    \item \textbf{Poe} est un chatbot qui fonctionne à partir de textes, offrant des interactions conversationnelles pour répondre aux questions des utilisateurs et fournir une assistance. Le logo de Poe est illustré à la figure \ref{fig:poe-logo}, permet de :
    \begin{figure}[ht]
        \centering
        \includegraphics[width=0.2\textwidth]{images/poe-logo.png}
        \caption{Logo de « poe »}
        \label{fig:poe-logo}
    \end{figure}
    \begin{itemize}[itemsep=1pt, parsep=1pt]
        \item Résoudre des problèmes en décrivant votre contexte. 
        \item Obtenir des réponses personnalisées à vos besoins. 
    \end{itemize}
\end{itemize}
