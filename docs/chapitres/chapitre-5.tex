\vfill 
\chapter{Réalisation}
\label{chap:conception}
\vfill 
\minitoc
\mtcaddchapter
\vfill 

\newpage

\section*{Introduction}
\justifying
Dans ce chapitre final, nous allons présenter les divers outils et framework employés pour la création de notre application. Aussi, nous présenterons également quelques interfaces de notre application.

\section{Environnement et outils de travail}
\justifying
Dans cette section, nous allons présenter les différents outils matériels et logiciels utilisés pour la mise en œuvre de notre application.
\subsection{Environnement matériel }
Pour mettre en place notre solution, nous avons utilisé deux ordinateurs portables. Le tableau 5.1 illustre leurs caractéristiques.
\begin{longtable}{|c|c|c|}
    \caption{Caractéristiques de l’environnement matériel} \\
    \hline
    & \textbf{Ordinateur 1} & \textbf{Ordinateur 2} \\
    \hline
    \textbf{Marque} & HP & Lenovo \\
    \hline
    \textbf{Processeur} & Intel i7 10\textsuperscript{ème} génération & Intel i5 8\textsuperscript{ème} génération \\
    \hline
    \textbf{Ram} & 12 GO & 8 GO \\
    \hline
    \textbf{Disque Dur} & 512 GO SSD & 512 GO SSD \\
    \hline
    \textbf{Système d’exploitation} & Windows 10 & Kubuntu LTS 22 \\
    \hline
\end{longtable}

\subsection{Environnement logiciel}
Le tableau 5.2 illustre la liste des outils utilisés lors du développement de notre application web. \newpage
\begin{longtable}{|p{4cm}|p{11cm}|}
    \caption{Liste des outils utilisé lors du développement de l’application} \\
    \hline
        \textbf{Outil/Technologie} & \textbf{Description} \\
    \hline
    \centering \textbf{Visual Studio Code} \vspace{0.2cm} \newline \centering \includegraphics[width=1.5cm,height=1.5cm]{chp5/vscode.png} & Visual Studio Code est un éditeur de code source développé par Microsoft reconnu pour sa légèreté, sa robustesse et ses extensions. \\
    \hline
    \centering \textbf{Postman} \vspace{0.2cm} \newline \centering \includegraphics[width=1.5cm,height=1.5cm]{chp5/postman.png} & Postman est une plateforme de développement API qui permet de créer, tester et déboguer des API de manière efficace. \\
    \hline
    \centering \textbf{Git} \vspace{0.2cm} \newline \centering \includegraphics[width=1.5cm,height=1.5cm]{chp5/git.png} & Git est un système de contrôle de version distribué et largement utilisé pour suivre les changements dans le code source. \\
    \hline
    \centering \textbf{GitHub} \vspace{0.2cm} \newline \centering \includegraphics[width=1.5cm,height=1.5cm]{chp5/github.png} & GitHub est une plateforme de développement logiciel basée sur Git qui offre des fonctionnalités de collaboration et de gestion de projets. \\
    \hline
    \centering \textbf{Draw.io} \vspace{0.2cm} \newline \centering \includegraphics[width=1.5cm,height=1.5cm]{chp5/drawio.png} & Draw.io est un outil de création de diagrammes en ligne qui permet de créer des diagrammes de manière intuitive et collaborative. \\
    \hline
    \centering \textbf{Excalidraw} \vspace{0.2cm} \newline \centering \includegraphics[width=1.5cm,height=1.5cm]{chp5/excalidraw.png} & Excalidraw est un outil de prototypage de l'interface utilisateur en ligne qui permet de créer des wireframes de manière simple et rapide. \\
    \hline
    \centering \textbf{React.js} \vspace{0.2cm} \newline \centering \includegraphics[width=1.5cm,height=1.5cm]{chp5/reactjs.png} & React.js est une bibliothèque JavaScript pour la création d'interfaces utilisateur interactives et dynamiques. \\
    \hline
    \centering \textbf{Next.js} \vspace{0.2cm} \newline \centering \includegraphics[width=2.5cm,height=1.5cm]{chp5/nextjs.png} & Next.js est un framework JavaScript React qui permet de construire des applications web performantes avec une expérience de développement simplifiée. \\
    \hline
    \centering \textbf{Tailwind CSS} \vspace{0.2cm} \newline \centering \includegraphics[width=1.5cm,height=1cm]{chp5/tailwind.png} & Tailwind CSS est une bibliothèque CSS utilitaire qui permet de concevoir rapidement des interfaces utilisateur modernes et personnalisées. \\
    \hline
    \centering \textbf{Prisma} \vspace{0.2cm} \newline \centering \includegraphics[width=1.5cm,height=1.5cm]{chp5/prisma.png} & Prisma est un ORM (Object-Relational Mapping) qui facilite l'interaction avec la base de données en utilisant un langage de requête TypeScript sécurisé. \\
    \hline
    \centering \textbf{PostgreSQL} \vspace{0.2cm} \newline \centering \includegraphics[width=1.5cm,height=1.5cm]{chp5/postgresql.png} & PostgreSQL est un système de gestion de base de données relationnelles robuste et performant. \\
    \hline
    \centering \textbf{Supabase} \vspace{0.2cm} \newline \centering \includegraphics[width=1.5cm,height=1.5cm]{chp5/supabase.png} & Supabase est une plateforme de développement intégrée qui offre une base de données PostgreSQL hébergée et d'autres fonctionnalités de backend. \\
    \hline
    \centering \textbf{Node.js} \vspace{0.2cm} \newline \centering \includegraphics[width=1.5cm,height=1.5cm]{chp5/nodejs.png} & Node.js est un environnement d'exécution JavaScript côté serveur qui permet d'exécuter du code JavaScript en dehors du navigateur. \\
    \hline
    \centering \textbf{TypeScript} \vspace{0.2cm} \newline \centering \includegraphics[width=1.5cm,height=1.5cm]{chp5/typescript.png} & TypeScript est un langage basé sur JavaScript développé par Microsoft avec un typage statique optionnel. Il facilite la détection précoce et la correction des erreurs lors du développement. \\
    \hline
    \centering \textbf{Vercel} \vspace{0.2cm} \newline \centering \includegraphics[width=1.5cm,height=1.5cm]{chp5/vercel.jpg} & Vercel est une plateforme de déploiement qui permet de déployer des applications frontend et backend de manière rapide, simple et évolutive. \\
    \hline
    \centering \textbf{Langchain} \vspace{0.2cm} \newline \centering \includegraphics[width=1.5cm,height=1.5cm]{chp5/langchain.png} & Langchain est un framework développé dans le but de faciliter la création d'applications en utilisant des modèles de grands modèles de langage (LLM). \\
    \hline
    \centering \textbf{Pinecone} \vspace{0.2cm} \newline \centering \includegraphics[width=1.5cm,height=1.5cm]{chp5/pinecone.png} & Pinecone est une base de données vectorielle qui offre une infrastructure efficace pour stocker et manipuler des données vectorielles. \\
    \hline
\end{longtable}

\section{Framework Next.js}
Notre application web est développée en utilisant le framework Next.js qui est construit sur la bibliothèque ReactJS et qui offre des fonctionnalités supplémentaires pour la création d'applications web modernes. Next.js est un framework full stack qui permet de créer des applications web performantes et optimisées en facilitant la création d'interfaces utilisateur.\\
Avec Next.js, nous pouvons utiliser le dossier \textbf{"app"} pour gérer le routage de notre application. Il permet de créer des routes dynamiques, des groupes de routes et des routes imbriquées en créant simplement des dossiers.\\
La Figure 5.1 illustre la structure du dossier \textbf{“app”}.\\


En outre, Next.js offre également un dossier spécial appelé \textbf{"api"} pour créer des endpoints API pour la partie backend de notre application.\\
Cela permet de créer facilement des API REST pour notre application.\\
De plus, il est important de noter que Next.js propose également les \textbf{"Server Actions"} qui peuvent également jouer un role rôle similaire à celui du dossier \textbf{"api"} pour créer des endpoints API.\\
Les Figures 5.2 et 5.3 illustrent la structure des dossiers \textbf{“api”} et \textbf{“actions”}.\\



Next.js simplifie ainsi la création d'applications web full stack en fournissant une solution complète pour la gestion du routage et de la création d'API.




\section{Implementation du Modèle LLM}
\subsection{API GroqCloud}
Dans notre plateforme, nous avons mis en œuvre le modèle Mixtrel of Experts (MoE) pour répondre à nos besoins en utilisant l'API de Groq, une société américaine spécialisée dans l'intelligence artificielle qui développe un accélérateur d'IA circuit intégré spécifique à l'application sous le nom de \textbf{L}anguage \textbf{P}rocessing \textbf{U}nit \textbf{(LPU)}. Cette API nous permet d'accéder aux différentes fonctionnalités du modèle MoE afin de créer un chatbot interactif capable de répondre aux questions des étudiants en se basant sur le contexte des documents partagés par eux-mêmes ou par leurs enseignants ainsi que sur des ressources externes. Cette fonctionnalité permettra aux étudiants d'obtenir des réponses rapides et précises à leurs questions et d'améliorer leur expérience utilisateur sur notre plateforme.\\
La figure 5.4 illustre l'intégration de l'API de Groq dans notre plateforme.\\